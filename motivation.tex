\section{Motivation}

Gas centrifuge cascades are usually designed to operate in an ideal manner, with
no losses in separative work. To achieve such ideal configuration, the cascade
is designed to be fed with a specific feed assay and produce the target
enrichment while rejecting tails at a fixed assay.

With the current international tensions regarding enrichment capabilities, this
work aims to measure the effectiveness of a symmetric enrichment cascade when
used outside of its designed scope and to quantify the attractiveness %Katie: I would be VERY VERY careful using the word attractiveness.
of such ways
to build up a \gls{SQ} of \gls{HEU}, defined by the \gls{IAEA} as \(25 \ kilograms\)\cite{noauthor_iaea_2001}.

The present work investigates the performance of an enrichment cycle when chaining
gas centrifuge enrichment cascades tuned for \gls{LEU} production from natural uranium to instead produce \gls{HEU}. Literature on the subject is limited due to its
internationally and politically sensitive nature. Three behavior models have been
implemented and used to evaluate the response of an enrichment cascade when fed
with different assays than originally designed. This work also takes advantage
of the \Cyclus \cite{cyclus} fuel cycle simulator's capabilities to evaluate the
assay values at equilibrium.
