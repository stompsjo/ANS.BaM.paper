\section{theory}
\subsection{Centrifuge properties}

The present work uses the analytical solution by R\"aetz \cite{raetz.phd} of the
differential equation for the gas centrifuge as described in \cite{glaser.2008}.
Centrifuge parameters, such as average gas temperature, $T$, peripheral speed,
$v$, height, $h$, diameter, $d$, pressure ratio, $x$, feed flow rate, $F$,
counter-current flow ratio, $L/P$, and efficiency, $e$ have been chosen (Table
\ref{tab:centrifuges}) to match the cascade design describe in
\cite{glaser.2008} and \cite{walker.2017}.  These parameters for a P1-type
centrifuge are used to estimate the JCPOA-compliant IR-1 centrifuge.

\begin{table}[htb]
\centering
\caption{Summary of the centrifuge parameters.}
\begin{tabular}{cccccccc}
\toprule
$T$[K] & $v$[m/s]    & $h$[m] & $d$[m]   & $x$   & $F$[mg/s]  & $L/F$ & $e$  \\
\midrule
320    & $320$           & $1.8$ & $0.105$ & $1e3$  & $13$      & 2     & 1.0  \\
\bottomrule
\end{tabular}

  \label{tab:centrifuges}
\end{table}

\subsection{Cascade Design}

The cascade is built as an ideal cascade, with no losses in the separative
work, which corresponds to $\alpha =\beta = const$ for all
stage of the cascade, where $\alpha$ and $\beta$ respectively represent the feed
to product and the feed to tail enrichment factors.  $\alpha$ and $\beta$ can be
expressed as function of the abondance ($R$) or the enrichment ($N$) of respectively the
product ($R'$,$N'$) and the feed,($R$,$N$) and the feed and the tails ($R''$,$N''$) such as:
\begin{subequations} \label{eqs:alphabeta}
    \begin{equation} \label{eq:alpha_def}
        \alpha = \frac{R'}{R} = \frac{N'}{1-N'}\frac{1-N}{N} 
\end{equation}
\\
\begin{equation}\label{eq:beta-def}
        \beta = \frac{R}{R''} = \frac{N}{1-N}\frac{1-N''}{N''} 
\end{equation}
\end{subequations}

As detailled in \cite{avery} it is also possible to derive $\alpha$ from the
first principle, and express it as a function of the feed rate $F$ the separative performance $\delta
U(\theta)$, the cut $\theta$:

    \begin{equation} \label{eq:alpha}
    \alpha = \sqrt{\frac{2\delta U}{F} \frac{1-\theta}{\theta}}+1
\end{equation}

From the mass conservation, $N = \theta N' + (1-\theta)N''$, and equations \eqref{eqs:alphabeta} it is
possible to express $\beta$ as a function of the feed abondance, $R$, the cut
$\theta$ and $\alpha$:

\begin{equation}\label{eq:beta}
    \beta =   R \left(\dfrac{1-\theta}
                     {\dfrac{R}{R+1}- \theta \dfrac{\alpha R}{1+\alpha R}} -1\right)
\end{equation}


From equation \eqref{eq:alpha} and \eqref{eq:beta} it is possible to determine
the cut, $\theta$, or the ratio of product flow to feed flow required to build an ideal
cascade:
$\beta$ values and the feed assay, $N_{i}$:
\begin{eqnarray}
    \theta_{i} = \dfrac{N_{i} - \dfrac{1}{1 + \beta/R_{i}}}{ \dfrac{\alpha R_{i}}{1 + \alpha R_{i}} -
           \dfrac{1}{1 + \beta/R_{i}}}
\end{eqnarray}

Since $\alpha_{i}$ and $\beta_{i}$ remain constant, only the value of the cut,
$\theta_{i}$, changes in each stage $i$ of a cascade.  This algorithm assumes
that the corresponding separative power $\delta U$ (not re-computed) can be
achieved with the chosen centrifuge design, tuning other operationnal paramter
such as the rotation speed, the counter-current flow ratio...  Once $\theta_{i}$
is determined, it is possible to compute the product and the tail assay.

The design of the cascade is performed through 2 steps.  First one determines the
configuration and number of stages, adding stages until the product assay of the
final stage is greater than or equal to than the desired assay, and the tails
assay is similarly less than or equal to the desired tails assay.  This
determines the number of enriching and stripping stages as well as their
enrichment properties ($N_{i}$, $N'_{i}$, $N''_{i}$,$\theta_{i}$i).


The second step determines how to populate the cascade with the user-defined
maximum number of centrifuges.  

One solves the linear flow equation, \eqref{eq:flow}, to determine the
theoretical flow in the cascade.

\begin{equation}
\setcounter{MaxMatrixCols}{20}
\begin{bmatrix}
     -1        & 1-\theta_{_{S+1}} & 0 & ...  & 0              & 0              & 0                 & 0                 & 0                 & ...  & 0               & 0 \\
 \theta_{_{S}} & -1                & 0 & ...  & 0              & 0              & 0                 & 0                 & 0                 & ...  & 0               & 0 \\
               &                   &   &     &                &                & ...               &                   &                   &     &                 &   \\
 0             & 0                 & 0 & ...  & \theta_{_{-2}} & -1             & 1 - \theta_{_{0}} & 0                 & 0                 & ...  & 0               & 0 \\
 0             & 0                 & 0 & ...  & 0              & \theta_{_{-1}} & -1                & 1 - \theta_{_{1}} & 0                 & ...  & 0               & 0 \\
 0             & 0                 & 0 & ...  & 0              & 0              & \theta_{_{0}}     & -1                & 1 - \theta_{_{2}} & 0   & ...             & 0 \\
               &                   &   &     &                &                & ...               &                   &                   &     &                 &   \\
 0             & 0                 & 0 & ...  & 0              & 0              & 0                 & 0                 & 0                 & ...  & -1              & 1-\theta_{_{E}} \\
 0             & 0                 & 0 & ...  & 0              & 0              & 0                 & 0                 & 0                 & ...  & \theta_{_{E-1}} & -1
 \end{bmatrix}
 \times
 \begin{bmatrix}
     F_{_{S}}   \\
     F_{_{S+1}} \\
     \cdots     \\
     F_{_{-1}}  \\
     F_{_{0} }  \\
     F_{_{1} }  \\
     \cdots     \\
     F_{_{E-1}} \\
     F_{_{E}}
 \end{bmatrix}
 =
 \begin{bmatrix}
     0      \\
     0      \\
     \cdots \\
     0      \\
     F      \\
     0      \\
     \cdots \\
     0      \\
     0
\end{bmatrix}
%\caption{caption needed!}
\label{eq:flow}
\end{equation}

Once the relative flow of each stage has been determined, the cascade can be
populated with actual machines up the stages until the maximum number available
of machines is reached.


\subsection{Miss-use models}
Little information is available about optimising an existing enrichment cascade
that is being fed with a feed enrichment that does not match the design one. So
far 3 different methods have been investigated, the first one assumes that no
change are been made on the cascade, the second one, assumes that the cut value
at each stage is retuned to maintain the ideal state of the cascade, the last
one, described in \cite{walker.2017} assumes the tails to product enriching
factor remains constants ($\gamma = \alpha\times\beta$). Models behavior, and
assumptions are summarized in Tab. \ref{tab:models}.




