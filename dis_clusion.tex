\section{Discussion}

It is really interesting to observe that when the cascade is left completely
untouched (Model A) or when it is slightly retuned to maintain the tail to
product enrichment factor as well as the cut of each centrifuges, chaining the
cascade we can observe high increase of the enrichment at each level.  On the
contrary, when retuning the cut of each centrifuges to maintain the ideal state
of the cascades while chaining them, the \gls{HEU} production rate is favored
over the enrichment gain.

The tail recycling allows for each model a hudge gain in productivity, even for
then model B for which it does not change the number of levels required to reach
$90w\%$ of $^{235}$U in the uranium. Even if no cascades chaining also to
retrieve the same production rate as a direct enrichment, the model B with
reprocessing reach about $80\%$ of an optimum production, which is far from
being negligible\ldots


\section{Conclusion}

This works has investigated and quantify the difference between potential
retuning of a gaseous enrichment cascade in order to chain them to produce
\gls{HEU} initially tuned to produce uranium enrichment for commercial reactors.
One of this tuning method allows up to $80\%$ of the production rate of a single
large enrichment cascade designed specifically for \gls{HEU} production using
the same number of centrifuges.

This works will be extended to the near future with additional miss-use method,
allowing for example the reconfiguration of the centrifuges in the cascades.

For this study, the usage of the Cyclus fuel cycle simulator was not really
required, it only allows a quick determination of the blending equilibrium. It
is planned to make use of the Cyclus Dynamical Resource Exchange full capability
in order to automatically assign the different cascades to the different level
as function of the resources availability, optimising the productions rates in
each cases.

While mathematically correct, the authors do not guaranty the feasibility
different miss-use tuning methods implemented and are welcoming any insight on
the matter.


