\documentclass{anstrans}
%%%%%%%%%%%%%%%%%%%%%%%%%%%%%%%%%%%
\title{Evidence of the necissity of isotopic precise calculation in fuel cycle
calculation}
\author{Baptiste Mouginot$^{*}$ Paul, P.H. Wilson$^{*}$}

\institute{
$^{*}$University of Wisconsin-Madison, WI
\and
%$^{\dagger}$State Capitol Building, Springfield, IL
}

\email{mouginot@wisc.edu \and paul.wilson@wisc.edu}

% Optional disclaimer: remove this command to hide
\disclaimer{Notice: this manuscript is a work of fiction. Any resemblance to
actual articles, living or dead, is purely coincidental.}

%%%% packages and definitions (optional)
\usepackage{graphicx} % allows inclusion of graphics
\usepackage{booktabs} % nice rules (thick lines) for tables
\usepackage{microtype} % improves typography for PDF

\newcommand{\SN}{S$_N$}
\renewcommand{\vec}[1]{\bm{#1}} %vector is bold italic
\newcommand{\vd}{\bm{\cdot}} % slightly bold vector dot
\newcommand{\grad}{\vec{\nabla}} % gradient
\newcommand{\ud}{\mathop{}\!\mathrm{d}} % upright derivative symbol

\begin{document}
%%%%%%%%%%%%%%%%%%%%%%%%%%%%%%%%%%%%%%%%%%%%%%%%%%%%%%%%%%%%%%%%%%%%%%%%%%%%%%%%
\section{Introduction} 

This work is part of the evaluation study of the transition analysis
\cite{Bo-fengpaper?} identify as the ``Evaluation Group 29'' or ``EG29'', in the
recently-published Evaluation and Screening (E\&S) \cite{ES}.  The EG29 fuel
cycle corresponds to a double strata fleet. The first stratum is composed by
sodium-cooled fast reactor (SFRs) multi-recycling the plutonium. The surplus
amounts of plutonium is send to the second stratum composed of pressurized water
reactor with full mixed oxide (U/Pu)O$_{2}$ cores (MOX-PWRs).  The PWR stratum
is also multi recycling its plutonium. The target energy generation ratio for
the final EG29 system is 70:30 (SFR:PWR), with a 1\% annual growth in nuclear
energy demand.

The purpose of this study is to enlighten the importance of the isotopic
composition for the plutonium, on PWR MOX fuel fabrication and the effect of its
evolution. Two main effect can modify the isotopic composition of the plutonium
used to build the PWR-MOX fuel: decay and depletion.


\section{Study description}
This works compares the plutonium amount require to build the fresh MOX fuel
using 3 different fabrication modeling: ``fixed mixing ratio'', ``plutonium
equivalent theory'', ``Neural network \cite{CLASS_MLP}'', considering or not
decay, and using recipe reactor or recalculation the depletion for each loaded
fuel.

The first part of the study in focused on the effect of the decay on the
isotopic composition of the reprocessed plutonium and its implication of fuel
fabrication depending on the fabrication modeling.
The second part of the study, compares 2 recipes compositions versus recalculated
depletion using the neural network fuel fabrication modeling. 

\subsection{Cycle}
The cycle used for the following study is a very simple cycle implying a single
PWR reactor using MOX fuel. After irradiation the uranium and the plutonium are
separated and re-processed to fuel the next batch of MOX fuel for the PWR.

A stream of ``good'' quality plutonium coming from an hypothetical irradiated
SFR blanket is represented by a storage. The impact of the decay on this storage
is limited, as the content in $^{241}$Pu is almost negligible.  In order to
build the first batch of fuel, an initial storage has also been set.  In order
to limit the effect of decay of this storage the initial inventory have been
tuned to be as small as possible (preventing any unnecessary plutonium pilling
up).

\subsection{Fabrication model}
Three different way to model the fuel fabrication have been used: ``fix mixing
ratio'', ``Pu-equivalent`` and ''Neural Network``.
The fix mixing ratio used pre-set ratio for the different stream mixing. The
''Pu-equivalent`` model try to fix the best mixing ratio allowing to reproduce
the initial reactivity of the requested MOX fuel. As the requested fuel
composition correspond to the fuel composition built with the ''fix mixing
ratio`` model, the results of those 2 models are expected to be close.
The last model use a different approach to mix the different stream: the
$<k_{infty}>$, the mean infinity neutronic multiplication factor, have to reach
a threshold at the end of the irradiation ($50~$GWd/t in this study). The
$<k_{infty}>$ used in this study is 1.034 \cite{???}. The Neural Network in this
model have been pre-trained on several different depletion calculation,
allowing it to predict the correct plutonium enrichment required.

Because of all those model have not necessary been tuned on exactly the same
PWR configuration and parameter, only the behavior differences between the
different calculations should be analysed..


%%%%%%%%%%%%%%%%%%%%%%%%%%%%%%%%%%%%%%%%%%%%%%%%%%%%%%%%%%%%%%%%%%%%%%%%%%%%%%%%
\section{Decay impact}

To measure the impact of the decay on the fuel fabrication process, $2\times3$
different calculations have been performed, each using a different way to model
the fuel fabrication, with and without decay activated:

\begin{itemize}
  \item fixed mixing ratio,
  \item plutonium equivalent theory,
  \item Neural network \cite{CLASS_MLP}.
\end{itemize}

The fixed mixing ratio corresponds to a simple mix of the different stream to
build the fresh MOX fuel. The mixing ratio remain fix regardless to the isotopic
composition evolution resulting of the decay process.

The second calculation uses the plutonium equivalent theory to determine the
proper mixing ratio between the different stream. In this case the model is
tuned to build a MOX fuel with the same initial reactivity as the fuel loaded
with the fixed mixing ratio model. 

The last model based on pre-trained neural network, determine the mixing ratio
in order to build a MOX fuel that will reach a $<k_{\infty}>$ of $1.034$ at the
end of irradiation (i.e.: $50~$GWd/t). (The $<k_{\infty}>$ corresponds to the mean
$k_{\infty}$ of the tree batches of the reactor) 

In order to discriminate between the depletion calculation effect and the decay
consequences, in each case the reactor use the same output composition for the
spent fuel despite the change in fresh MOX fuel composition.



%%%%%%%%%%%%%%%%%%%%%%%%%%%%%%%%%%%%%%%%%%%%%%%%%%%%%%%%%%%%%%%%%%%%%%%%%%%%%%%%
\section{Irradiation calculation impact}

This part is dedicated to the evaluation of the impact of non recalculation of
the MOX fuel evolution during irradiation.

Two additional calculation have been performed. Those calculations use the
neural network for the fuel fabrication modeling as well as for the
recalculation of the MOX fuel depletion calculation: for each MOX fuel loaded in
the reactor, a dedicated depletion calculation also to update the output recipe
used for the spent MOX fuel.


The difference between the fix recipe reactor calculation and the more precise
reactor are show in the following part.


%%%%%%%%%%%%%%%%%%%%%%%%%%%%%%%%%%%%%%%%%%%%%%%%%%%%%%%%%%%%%%%%%%%%%%%%%%%%%%%%
\section{Conclusions}

The included ANS style file and this clear example file are a panacea for
the hours of headache that invariably results from formatting a document in
Microsoft Word.

%%%%%%%%%%%%%%%%%%%%%%%%%%%%%%%%%%%%%%%%%%%%%%%%%%%%%%%%%%%%%%%%%%%%%%%%%%%%%%%%
\appendix
\section{Appendix}

Numbering in the appendix is different:
\begin{equation} \label{eq:appendix}
  2 + 2 = 5\,.
\end{equation}
and another equation:
\begin{equation} \label{eq:appendix2}
  a + b = c\,.
\end{equation}

%%%%%%%%%%%%%%%%%%%%%%%%%%%%%%%%%%%%%%%%%%%%%%%%%%%%%%%%%%%%%%%%%%%%%%%%%%%%%%%%
\section{Acknowledgments}
This material is based upon work supported a Department of Energy Nuclear
Energy University Programs Graduate Fellowship.

%%%%%%%%%%%%%%%%%%%%%%%%%%%%%%%%%%%%%%%%%%%%%%%%%%%%%%%%%%%%%%%%%%%%%%%%%%%%%%%%
\bibliographystyle{ans}
\bibliography{bibliography}
\end{document}

