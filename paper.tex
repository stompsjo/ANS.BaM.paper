\documentclass{anstrans}
%%%%%%%%%%%%%%%%%%%%%%%%%%%%%%%%%%%
\title{An Improved ANS Transaction Template}
\author{Baptiste Mouginot,$^{*}$ Kathryn Mummah,$^{*}$ Paul P.H. Wilson$^{*}$}

\institute{
$^{*}$University of Wisconsin-Madison, WI
}

\email{mouginot@wisc.edu \and mummah@wisc.edu \and paul.wilson@wisc.edu}

% Optional disclaimer: remove this command to hide
% \disclaimer{Notice: this manuscript is a work of fiction. Any resemblance to
% actual articles, living or dead, is purely coincidental.}

%%%% packages and definitions (optional)
\usepackage{graphicx} % allows inclusion of graphics
\usepackage{booktabs} % nice rules (thick lines) for tables
\usepackage{microtype} % improves typography for PDF
\usepackage{float}

\newcommand{\SN}{S$_N$}
\renewcommand{\vec}[1]{\bm{#1}} %vector is bold italic
\newcommand{\vd}{\bm{\cdot}} % slightly bold vector dot
\newcommand{\grad}{\vec{\nabla}} % gradient
\newcommand{\ud}{\mathop{}\!\mathrm{d}} % upright derivative symbol

\begin{document}
%%%%%%%%%%%%%%%%%%%%%%%%%%%%%%%%%%%%%%%%%%%%%%%%%%%%%%%%%%%%%%%%%%%%%%%%%%%%%%%%
\section{Introduction}

The Joint Comprehensive Plan of Action (JCPOA)\cite{ref} %cite JCPOA agreement
limits the Iranian nuclear program and, in return, lifts many of the economic
sanctions that had been placed on Iran. Under the JCPOA, Iran is severely
limited in its ability to enrich Uranium. This paper presents a model to probe
the enriching ability of various cascade configurations that are related to
JCPOA agreement limits. This analysis was performed using the Cyclus fuel cycle
simulator\cite{ref} %cite Fundamental concepts in the Cyclus nuclear fuel cycle
simulation framework in order to demonstrate that such analysis can be
integrated into a fuel cycle with material flowing among multiple facilities.

The Cyclus fuel cycle simulator is designed to support the dynamic addition
novel facility models into fuel cycles both declared and undeclared facilities.
The CascadeEnrich archetype\cite{ref} was developed to model gas centrifuge
enrichment cascades at a higher fidelity than previous enrichment models,
relying on individual centrifuge and cascade design parameters to build an ideal
cascade configuration instead of integral facility parameters. Once a
CascadeEnrich facility is deployed, it may be asked to participate in the fuel
cycle at a variety of flow rates and enrichments over time, constrained by the
initial design of the cascade.

This analysis introduces a facility with parameters consistent with the
JCPOA-defined capacity and explores its performance when asked to enrich
material above JCPOA limits. Although the analytical aspect of this analysis can
be performed using other tools, a fuel cycle simulator provides value in further
extensions of this work. Studying the internal behavior of a cascade, or rather
a set of cascades organized into levels that feed into each other, identifies
the most effective ways that enrichments higher than the JCPOA-allowed 3.67\%
U-235 could be produced. However, this is only one step in a safeguards program.

Performing enrichment cascade analysis within a fuel cycle simulator offers the
opportunity to study the \textit{external} behavior of an enrichment facility as
well as internal. When an enrichment facility that is not complying with the
JCPOA is forced to participate in a larger fuel cycle---where it must purchase
natural uranium and interact with outside facilities, it will act differently
than an enrichment facility that \textit{is} fully complying with the JCPOA. The
levels of cascades developed in this paper will ultimately be used to probe
these differences and reveal patterns useful for a safeguards program.


\section{Cascade Enrich Construction}
\subsection{Centrifuge properties}

The present work uses the analytical solution by R\"aetz \cite{ref} of the
differential equation for the gas centrifuge as described in \cite{Glaser2008}.
Centrifuge parameters, such as average gas temperature, rotation speed, height,
feed flow rate, diameter, pressure ratio, counter-current flow ratio, and
efficiency have been chosen to match the cascade design describe in
\cite{glaser2008} and \cite{Walker2017}. These parameters for a P1-type
centrifuge are used to estimate the JCPOA-compliant IR-1 centrifuge.


\subsection{Cascade Design}

The cascade is built to be ideal, defined by $\alpha =\beta = const$ for all
stage of the cascade, where $\alpha$ and $\beta$ respectively represent the feed
to product and the feed to tail enrichment factors.  $\alpha$ can be expressed
as a function of the feed rate $F$, the separative performance $\delta
U(\theta)$ and $\theta$, and $\beta$ as a function of $alpha$ and $\theta$:

\begin{subequations} \label{eqs:alphabeta}
    \begin{equation} \label{eq:alpha}
    \alpha = \sqrt{\frac{2\delta U}{F} \frac{1-\theta}{\theta}}+1
\end{equation}
\\
\begin{equation}\label{eq:beta}
    \beta = R
              \dfrac{1 - \dfrac{N - \theta \dfrac{\alpha R}{1+\alpha R}}{1 - \theta}}
                   {\dfrac{N - \theta \dfrac{\alpha R}{1+\alpha R}}{1 - \theta}}
\end{equation}
\end{subequations}

From equation \eqref{eq:alpha} and \eqref{eq:beta} it is possible to determine
the cut, or the ratio of product flow to feed flow.
%(all other centrifuge parameters been fixed) allowing the have $\alpha=\beta$
required to build an ideal cascade.

Since $\alpha_{i}$ and $\beta_{i}$ remain constant, only the value of the cut,
$\theta_{i}$, changes in each stage of a cascade.

It can be shown that $\theta_{i}$ can be computed from $\alpha$,
$\beta$ values and the feed assay, $N_{i}$ \cite{ref}:
\begin{eqnarray}
    \theta_{i} &=& \dfrac{N_{i} - N''_{i}}{N'_{i}-N''_{i}}\nonumber\\
           &=& \dfrac{N_{i} - \dfrac{1}{1 + \beta/R_{i}}}{ \dfrac{\alpha R_{i}}{1 + \alpha R_{i}} -
           \dfrac{1}{1 + \beta/R_{i}}}
\end{eqnarray}

This algorithm assumes that the corresponding separative power $\delta U$ (not
re-computed) can be achieved with the chosen centrifuge design. Once
$\theta_{i}$ determined, it is possible to compute the product and the tail
assay.

The design of the cascade is performed through 2 steps. First one determines the
configuration and number of stages, adding stages following the previously
described procedure, until $N'_{last} > N'_{requested}$ and $N''_{last} <
N''_{requested}$. (Note that $N'_{last}$ corresponds to the product assay of the
last enrichment stage, and $N''_{last}$ to the tail assay of the last stripping
stage). This allows to determine the number of enriching and stripping stages as
well as their enrichment properties ($N_{i}$, $N'_{i}$,
$N''_{i}$,$\theta_{i}$i).
% might be more useful to the reader to say this in words instead of variables,
% i.e. adding stages until the product assay of the final stage is greater than
% or equal to than the desired assay, and the tails assay is similarly less than
% or equal to the desired tails assay

The second step determines how to populate the cascade with the user-defined
maximum number of centrifuges. One first solves the linear flow equation to
determine the theoretical flow in the cascade:

\begin{strip}
\begin{equation}
\setcounter{MaxMatrixCols}{20}
\begin{bmatrix}
-1         & 1-\theta_{S+1} & 0              & 0 & ... & 0 & 0            & 0   & 0              & 0 & ... & 0            & 0            & 0 \\
\theta_{S} & -1             & 1-\theta_{S+2} & 0 & ... & 0 & 0            & 0   & 0              & 0 & ... & 0            & 0            & 0 \\
           &                &                &   &     &   &              & ... &                &   &     &              &              &   \\
0          & 0              & 0              & 0 & ... & 0 & \theta_{-1}  & -1  & 1 - \theta_{1} & 0 & ... & 0            & 0            & 0 \\
           &                &                &   &     &   &              & ... &                &   &     &              &              &   \\
0          & 0              & 0              & 0 & ... & 0 & 0            & 0   & 0              & 0 & ... & \theta_{E-2} & -1           & 1-\theta_{E} \\
0          & 0              & 0              & 0 & ... & 0 & 0            & 0   & 0              & 0 & ... & 0            & \theta_{E-1} & -1
\end{bmatrix}
\times
\begin{bmatrix}
     F_{S}   \\
     F_{S+1} \\
     \cdots  \\
     F_{0}   \\
    \cdots   \\
    F_{E-1}  \\
    F_{E}
\end{bmatrix}
=
\begin{bmatrix}
     0   \\
     0 \\
     \cdots  \\
     F   \\
    \cdots   \\
    0  \\
    0
\end{bmatrix}
%\caption{caption needed!}
\label{eq:flow}
\end{equation}
\end{strip}
Once the relative flow of each stage, $F_{i}$, has been determined, the
cascade can be populated with actual machines up the stages
until either the maximum number available of machines or the maximum feed
flow is reached.


\subsection{Response to an non-ideal feed - $\theta_{i} = const$ hypothesis}

Little information is available about the optimum way to tune a cascade that is
being fed a feed enrichment that does not match its ideal enrichment.

The tuning method outlined here does not re-optimize $\theta_i$ based on the
true flow enrichment. As $\delta U$ and $\alpha$ do not depend on the stage feed
assay $N'$ and the feed to product ratio $\alpha$ do not change from stage to
stage. According to equation \eqref{eq:beta}, when $\alpha$ and $\theta$ are
fixed, if the feed assay, $N$ changes, then $\beta$ will change accordingly.
This brakes the ideal status of the cascade, i.e.  $N_{i} \neq N'_{i-1} \neq
N''_{i+1}$.

In order to compute the proper product and tails assay at each stages\, the
tails and the product from respectively the next and the previous stage must be
blended in order to determine the correct stage feed assay. As this is a obvious
cycling problem, it has been chosen to solve it iteratively: all feed assay are
iteratively updated, blending the proper product and tail, then using the
updated feed assay, the new product and tail assay are recomputed. This process
is repeated until the change in assays is smaller than the set precision (1e-8
by default).

Other hypotheses will be explored in the future, such as maintaining the ideal
stage of the cascade through tuning the cut values $\theta_{i}$ of each stage of
the cascade, or maintaining $\alpha*\beta = const$ as described in
\cite{walker2017}.

Note that as a consequence of our design method, the cascade product and tail
assay will not necessary match the targeted values, and usually slightly over
enrich and over strip the product and the tail respectively.

\section{The experiment}

As explained previously, the goal of this work is to assess the rate of the
highly-enriched uranium (HEU) production using enrichment cascades designed
within the JCPOA agreement constraints. For this work, two different cases will
be investigated and compared to the reference case.

All cases will be considered with and without tail recycling. Figure
\ref{fig:flow} shows the different flows between the different facilities in the
fleet, in a configuration with two different levels of cascade without tail
recycling. When considering the tail recycling, all the flow going from a
cascade to the waste, are going from the cascade $i$ to the cascade $i-1$.  The
Storage are not required in Cyclus they have been added to ensure the proper
material flow between the facilities (if a receiver facility input buffer is
full then the sender facility will stop sending material to it, and will stop
working when its own output buffer will be filled).

\begin{figure}[ht] % replace 't' with 'b' to force it to be on the bottom
  \centering
  \includegraphics[scale=0.7]{flow_case_2_no_recy.pdf}
  \caption{Illustration of the material flow between the different level of
      cascades without tail recycling, in this diagram, nu\_src correspond to an
      infinite source of natural uranium, cascade\_ $n$ and storage\_ $n$
      corresponds respectively to the cascades, and the storage getting the
  products of the cascades at the level $n$.}\label{fig:flow}
\end{figure}

\subsection{Explored Cases}
\subsubsection{Reference Case}

The reference case corresponds to the most ideal cascade design--utilizing all
of the 5060 centrifuges available within JCPOA to produce an enriched product at
$>90\%$ of U-235, stripping the tail product up to 0.28 $\%$ from a feed of
natural uranium.  The design cascade in this case, includes 4 stripping stages,
and 38 enriching stages. The Feed/Product/Tail (F/P/T) assays are respectively
0.0071/0.903/0.0028.

\subsubsection{Case 1}

The case 1 is designed to configure 30 cascades in the scenario (the maximum
allowable under the JCPOA agreement), all designed the same way. The basic
properties are: less than 169 centrifuges per cascade, F/P/T assays of
0.0071/0.035/0.003. The corresponding design includes 167 centrifuges, with
respective F/P/T assays of 0.0071/0.0412842/0.00290548.

\subsubsection{Case 2} 

Case 2 also limited to 30 cascades in the scenario (the maximum allowable under
the JCPOA agreement), but this time each level of cascades are designed to take
feed from the previous level's product. Each cascade has 4 stripping stages and
10 enriching stages. 

\begin{table}[htb]
\centering
\begin{tabular}{cllll}
\toprule

Level   &           & Assay     &       & Machines  \\
        & Feed      & Product   & Feed  &           \\
\midrule
0       & 0.0071    & 0.0413    & 0.0029 & 167       \\
1       & 0.0413    & 0.2043    & 0.0173 & 169       \\
2       & 0.2043    & 0.5941    & 0.0971 & 168       \\
3       & 0.5941    & 0.8834    & 0.3915 & 168       \\
4       & 0.8834    & 0.9735    & 0.7746 & 169       \\

\bottomrule
\end{tabular}
  \caption{Summary of cascade properties for each level}
  \label{tab:cascadelvl}
\end{table}

The configuration of cascade levels can be found in table
\ref{tab:cascadelvl}. Unlike case 1, the organization of centrifuges varies from
level to level. Each level is re-optimized to have the most efficient number of
centrifuges in each stage, although the total number of stages remains constant.


\subsection{Level optimization}

A last round of optimization if then ran in order to limit the number of cascade
level to the amount of level required produce HEU, and then to populate the
different level with a total of 30 cascades in order to maximize the HEU
production rate.

\section{Results}
\subsection{Enrichment}
\subsubsection{Without tail recycling}

Despite the reconfiguration of each level occurring in the second case, the case
1 is notably more effective in producing HEU than case 1. Indeed, as observed in
Figure \ref{fig:assay_c1_rc} and \ref{fig:assay_c2_rc}, the first case is able
to produce an uranium product enrich at $98\%$ while the second case is only
able to reach an enrichment of $88.3\%$ with the same number of levels.

\begin{figure}[ht] % replace 't' with 'b' to force it to be on the bottom
  \centering
  \includegraphics[width=0.5\textwidth]{assay_case_1_no_rec.png}
  \caption{Evolution of the different product assays at each cascade level for
  the case 1 without recycling. The blue line represents the $90\%$ enrichment
  threshold.}\label{fig:assay_c1_nr}
\end{figure}
\begin{figure}[ht] % replace 't' with 'b' to force it to be on the bottom
  \centering
  \includegraphics[width=0.5\textwidth]{assay_case_2_no_rec.png}
  \caption{Evolution of the different product assays at each cascade level for
  the case 2 without recycling. }\label{fig:assay_c2_nr}
\end{figure}

Because the cut value $\theta_i$ at each stage is unchanged, case 2 artificially
produces over-enriched tails when fed with higher enriched materials compared
to a cascade feed with the same feed assay but with a cascade designed for it.

% ^ the above paragraph is confusing but I don't understand it enough to fix it

The effective cut values for each cascade can be computed as :

\begin{equation}\label{eq:theta_eff}
    \theta = \dfrac{N - N''}{N'-N''}
\end{equation}

\begin{table}[htb]
\centering
\begin{tabular}{cll}
\toprule

Level   &  $\theta_{eff}^{C1}$   & $\theta_{eff}^{C2}$ \\
\midrule
1       & 0.109375               & 0.109375     \\
2       & 0.109375               & 0.128342     \\
3       & 0.109375               & 0.215694     \\
4       & 0.109375               & 0.411872     \\
5       & 0.109375               & 0.547009     \\

\bottomrule
\end{tabular}
  \caption{Cascade effective $\theta$ by the level and case.}
  \label{tab:cascade_theta}
\end{table}
As illustrated in Table \ref{tab:cascade_theta}, the effective cut
of each level increases up to $0.54$ for the fifth stage for case 2. The high
effective cut value increase the product production rate but decrease the
enrichment ratio.

\subsubsection{tails recycling}
When recycling the tails from the different cascade levels, both cases behave
the same way. As the tails have an higher enrichment than the product then are
blended to, recycling the tail has the effect to bump up the enrichment of the
product at each level (see Figure \ref{fig:assay_c1_r} and \ref{fig:assay_c2_r}).

\begin{figure}[ht] % replace 't' with 'b' to force it to be on the bottom
  \centering
  \includegraphics[width=0.5\textwidth]{assay_case_1_rec.png}
  \caption{Evolution of the different product assays at each cascade level for
  the case 1 with recycling. }\label{fig:assay_c1_r}
\end{figure}

Note that for the case 2, where the fourth stage could appear as useless only the
blending of its tail allows the product of the third one to pass the $90\%$ threshold.

\begin{figure}[ht] % replace 't' with 'b' to force it to be on the bottom
  \centering
  \includegraphics[width=0.5\textwidth]{assay_case_2_rec.png}
  \caption{Evolution of the different product assays at each cascade level for
  the case 2 with recycling.}\label{fig:assay_c2_r}
\end{figure}


\subsection{HEU flow rate}
As shown Figure \ref{fig:heu_prod}, if using all the centrifuges available to
build a big cascade dedicated to HEU production, only 8 months are required to
produce 1 Significant Quantity (SQ) of HEU. Also it take significantly longer to
build 1 SQ of HEU, without recycling the tails respectively 13 years and 3.5 year
for the case 1 and 2 versus about 2 years with tail recycling.

\begin{figure}[ht] % replace 't' with 'b' to force it to be on the bottom
  \centering
  \includegraphics[width=0.5\textwidth]{HEU_prod.png}
  \caption{Evolution of the amount of HEU in the storage. The blue line
      correspond to one significant quantity of HEU ($25~$kg). CX correpsonds to
      the case number(1 or 2), RT and NRT the tail reprocessing or not ones.
      W/X/Y/Z represent the number of cascade per level. REF corresponds to the
      reference case.}\label{fig:heu_prod}
\end{figure}

It is also interresting to note that despite requiering more level of cascade to
reach an enrichment of $90\%$ the second case with tial recycling produces HEU
at a equivalent rate than the case 1. The production rate seems to be faster but
require more time to reach the maximum production rate (the material have to
flow through more levels to reach the final level).


%%%%%%%%%%%%%%%%%%%%%%%%%%%%%%%%%%%%%%%%%%%%%%%%%%%%%%%%%%%%%%%%%%%%%%%%%%%%%%%%
%%%%%%%%%%%%%%%%%%%%%%%%%%%%%%%%%%%%%%%%%%%%%%%%%%%%%%%%%%%%%%%%%%%%%%%%%%%%%%%%
\section{Conclusions}

The JCPOA aggrement limits to 5060 runing centrifuges in the Natanz Fuel
Enrichment Plant (NFEP) so about $10\%$ of its design capacity, the reference
case allows to build $1$ SQ of HEU in 8 months (versus in 24 days without the JCPOA
agrement limitations). While using directly the 5060 centrifuges to produce 1 SQ
of HEU take about 8 months, case 1 and the case 2 will produce it in 2 years.
Nevertheless the diversion scenario from case 1 or 2 will be hard if not
impossible to detect from the cascasde configuration.  

While the results of this work might be limited, it aims to develop a
methodology to study diversion scenarios, from the fuel cycle perspective.
Futhermore, additional methods to compute the CascadeEnrich response to
non-ideal feed assay will be investigated.  The CascadeEnrich facility used will
also be improved in order to take advantage of the dynamical flow managment of
the Cyclus framework, allowing for example to automatically assign the cascade
to the different level as the need evolve.



%%%%%%%%%%%%%%%%%%%%%%%%%%%%%%%%%%%%%%%%%%%%%%%%%%%%%%%%%%%%%%%%%%%%%%%%%%%%%%%%
%% \appendix
%% \section{Appendix}
%% 
%% Numbering in the appendix is different:
%% \begin{equation} \label{eq:appendix}
%%   2 + 2 = 5\,.
%% \end{equation}
%% and another equation:
%% \begin{equation} \label{eq:appendix2}
%%   a + b = c\,.
%% \end{equation}
%% 
%%%%%%%%%%%%%%%%%%%%%%%%%%%%%%%%%%%%%%%%%%%%%%%%%%%%%%%%%%%%%%%%%%%%%%%%%%%%%%%%
%% \section{Nomenclature}
%%
%% \begin{table}[H]
%%     \centering
%%     \begin{tabular}{l|l}
%% %         &  \\
%%         $N$ & Feed assay \\
%%         $N'$ & Product assay \\
%%         $N''$ & Tails assay \\
%%         $\alpha$ & Feed to product enrichment factor \\
%%         $\beta$ & Feed to tail enrichment factor \\
%%         $\theta$ & Cut
%%     \end{tabular}
%% %    \caption{Caption}
%%     \label{tab:my_label}
%% \end{table}

%%%%%%%%%%%%%%%%%%%%%%%%%%%%%%%%%%%%%%%%%%%%%%%%%%%%%%%%%%%%%%%%%%%%%%%%%%%%%%%%
\section{Acknowledgments} 
This work was funded by the Consortium for Verification Technology under
Department of Energy National Nuclear Security Administration award number
DE-NA0002534

%%%%%%%%%%%%%%%%%%%%%%%%%%%%%%%%%%%%%%%%%%%%%%%%%%%%%%%%%%%%%%%%%%%%%%%%%%%%%%%%
\bibliographystyle{ans} 
\bibliography{bibliography} 
\end{document}
